%%%%%%%%%%%%%%%%%%%%%%%%%%%%%%%%%%%%%%%%%%%%%%%%%%%%%%%%%%%%%%%%%%%%%%%%%%%%%%%%
%2345678901234567890123456789012345678901234567890123456789012345678901234567890
%        1         2         3         4         5         6         7         8

\documentclass[letterpaper, 10 pt, conference]{ieeeconf} 
\IEEEoverridecommandlockouts                
\overrideIEEEmargins

\title{\LARGE \bf Tres Estructuras de Datos Probabilistico
}

\author{Samuel Leiva$^{1}$, Junior Micha$^{2}$ , Danilo Blas$^{3}$ Joel Janampa$^{4}$, Brener Bustillos $^{5}$ % 
\thanks{Manuscrito creado el 11 de setiembre del 2018; cuya revisio\'n final sera el 363 de dicimebre .Este trabajo es compatible en formato IEEE y se distribuye bajo el Proyecto LaTeX.El manuscrito puede ser encontrado en los github de los autores}% <-this % stops a space
\thanks{$^{1}$ S.Leiva es estudiante de ciencias de la computacion 
        Universidad Nacional de Ingenieria,2015-2021, Lima,Peru.
        {\tt\small https://github.com/SamuelLeiva}}%
\thanks{$^{2}$ J.Micha es estudiante de pregrado de matematica ,Universidad Nacional de Ingenieria,2016-2022,Lima,Peru.
        {\tt\small https://github.com/JMicha23}}%
\thanks{$^{3}$ D.Blas es estudiante de ciencias de la computacion ,Universidad Nacional de Ingenieria,2015-2021,Lima,Peru.
        {\tt\small https://github.com/Sdann26}}%
\thanks{$^{4}$ J.Janampa es estudiante de matematica,Universidad Nacional de Ingenieria,2015-2021,Lima,Peru.
        {\tt\small https://github.com/JoelJanampaBautista}}%
\thanks{$^{5}$ B.Bustillos es estudiante de matematica,Universidad Nacional de Ingenieria,Universidad Nacional de Ingenieria,Lima,Peru. 
        {\tt\small https://github.com/brenner-08}}%
}


\begin{document}



\maketitle
\thispagestyle{empty}
\pagestyle{empty}


%%%%%%%%%%%%%%%%%%%%%%%%%%%%%%%%%%%%%%%%%%%%%%%%%%%%%%%%%%%%%%%%%%%%%%%%%%%%%%%%
\begin{abstract}
    

Este articulo matematico-computacional describe tres tipos de estructuras de datos probabilisticos ,como son Bloomfilter,Count-min Sketch y Hyperlog. Ademas de eso daremos algunas aplicaciones trabajando con el lenguaje de programacion R y verificando el funcionamiento para el conteo de las consultas referidas a elemnetos dentro de un conjunto cualquiera

\end{abstract}


%%%%%%%%%%%%%%%%%%%%%%%%%%%%%%%%%%%%%%%%%%%%%%%%%%%%%%%%%%%%%%%%%%%%%%%%%%%%%%%%
\section{INTRODUCCIO\'N}

usar la seccion para poner 

\section{RESUMEN}
Usar la estructura Bloom Filter para hacer consultas sobre elemntos que puedan pertenecer a la estructura.\\
usar la estructura count-min sketch 

\section{OBJETIVOS}


\subsection{JSJSJS}

\subsection{UnDSDFs}

\begin{itemize}

\item URWER 
\item AERWE
\item DFSDF
\item DDD, 

\end{itemize}


\subsection{Ecuaciones}

sdajflas ecuanciones

$$
\alpha + \beta = \chi \eqno{(1)}
$$

Note that the equation is centered using a 

\subsection{subitituo}
\begin{itemize}


\item 
\item 
\item  

\end{itemize}


\section{INTRODUCCION}

RELLENAR 

\subsection{PIRMER}



\subsection{LLENAR SI SE DESEA}



   \begin{figure}[thpb]
      \centering
      \framebox{\parbox{3in}{wiiiiisdshdfksbsdka
}}
      %\includegraphics[scale=1.0]{figurefile}
      \caption{descripcion de la imagen }
      \label{figurelabel}
   \end{figure}
   


\section{CONCLUSIONS}



\addtolength{\textheight}{-12cm}


%%%%%%%%%%%%%%%%%%%%%%%%%%%%%%%%%%%%%%%%%%%%%%%%%%%%%%%%%%%%%%%%%%%%%%%%%%%%%%%%

%%%%%%%%%%%%%%%%%%%%%%%%%%%%%%%%%%%%%%%%%%%%%%%%%%%%%%%%%%%%%%%%%%%%%%%%%%%%%%%%

%%%%%%%%%%%%%%%%%%%%%%%%%%%%%%%%%%%%%%%%%%%%%%%%%%%%%%%%%%%%%%%%%%%%%%%%%%%%%%%%
\section*{APPENDIX}

rellenear

\section*{ACKNOWLEDGMENT}

%%%%%%%%%%%%%%%%%%%%%%%%%%%%%%%%%%%%%%%%%%%%%%%%%%%%%%%%%%%%%%%%%%%%%%%%%%%%%%%%
\begin{thebibliography}{99}

\bibitem{c1}
\bibitem{c2} 
\bibitem{c3} 


\end{thebibliography}

\end{document}