%%%%%%%%%%%%%%%%%%%%%%%%%%%%%%%%%%%%%%%%%%%%%%%%%%%%%%%%%%%%%%%%%%%%%%%%%%%%%%%%
%2345678901234567890123456789012345678901234567890123456789012345678901234567890
%        1         2         3         4         5         6         7         8

\documentclass[letterpaper, 10 pt, conference]{ieeeconf}  

\IEEEoverridecommandlockouts                              

\overrideIEEEmargins


\title{\LARGE \bf
Tres Estructuras de Datos Probabili\'sticos
}



\author{Samuel Leyva$^{1}$ ,Junior Micha$^{2}$ ,Danilo Blas$^{3}$ ,Joel Janampa$^{4}$ y Brener Bustillos$^{5}$
\thanks{Manuscrito creado el 11 de setiembre del 2018; cuya revisio\'n final sera el 363 de dicimebre .Este trabajo es compatible en formato IEEE y se distribuye bajo el Proyecto LaTeX.\\El manuscrito puede ser encontrado en los github de los autores: \\ 
\author{Samuel leyva$^{1}$ ,Junior Micha$^{2}$, Danilo Blas$^{3}$ ,Joel Janampa$^{4}$ y Brener Bustillos$^{5}$n}% <-this % stops a space
\thanks{$^{1}$S. Leyva es estudiate de Ciencia de la Computacion ,de la Universidad Nacional de Ingenieria ,Lima,Peru.
        {\tt\small smleiva-github}}%
\thanks{$^{2}$J. Micha es estudiante de matematica, Universidad Nacional de Ingenieria ,
        Lima, Peru
        {\tt\small j.micha23-github}}%
\thanks{$^{3}$D. Blas es estudiate de Ciencia de la Computacion ,de la Universidad Nacional de Ingenieria ,Lima,Peru.
        {\tt\small D.blas45-github}}%
\thanks{$^{4}$J. Janampa es estudiante de matematica, Universidad Nacional de Ingenieria ,
        Lima, Peru
        {\tt\small janampa69-github}}%
\thanks{$^{5}$P. B. Bustillos es estudiate de matematica ,de la Universidad Nacional de Ingenieria ,Lima,Peru.
        {\tt\small b.bustillos654-github}}%                        
}


\begin{document}



\maketitle
\thispagestyle{empty}
\pagestyle{empty}


%%%%%%%%%%%%%%%%%%%%%%%%%%%%%%%%%%%%%%%%%%%%%%%%%%%%%%%%%%%%%%%%%%%%%%%%%%%%%%%%
\begin{abstract}

Este articulo matematico-computacional describe tres tipos de estructuras de datos probabilisticos ,como son Bloomfilter,Count-min Sketch y Hyperlog. Ademas de eso daremos algunas aplicaciones trabajando con el lenguaje de programacion R y verificando el funcionamiento para el conteo de las consultas referidas a elemnetos dentro de un conjunto cualquiera.

\end{abstract}


%%%%%%%%%%%%%%%%%%%%%%%%%%%%%%%%%%%%%%%%%%%%%%%%%%%%%%%%%%%%%%%%%%%%%%%%%%%%%%%%
\section{INTRODUCTION}


\section{INTRODUCCIO\'N}

\subsection{Objetivos}

usar la estructura bloom filter para hacer consultas sobre elemntos que pueddan pertencer a la estructura\\
más cosas
\subsection{subtitulo}



\section{MATH}



final complementario
\subsection{Abbreviations and Acronyms}
ahaegdbcjdbe

\subsection{Units}

\begin{itemize}

\item 
\item 
\item 



\subsection{Equations}



$$
\alpha + \beta = \chi \eqno{(1)}
$$

notaciones o no se

\subsection{Some Common Mistakes}
\begin{itemize}



\item 
\item
\item 
\item 
\item 

\end{itemize}


\section{usae el titulo o subtirtulo relenar}

kkjwejfifbisdbfsbrifubwiednsmdcsmdclñsmdñlcmskldnvlksnrjovs

\subsection{subitutlo 2, etc}

poner no se que

\subsection{Figures and Tables}

posicion de las figuras

\begin{table}[h]
\caption{ejmeplo de tabla}
\label{table_example}
\begin{center}
\begin{tabular}{|c||c|}
\hline
uno & doa\\
\hline
Three & Four\\
\hline
\end{tabular}
\end{center}
\end{table}


   \begin{figure}[thpb]
      \centering
      \framebox{\parbox{3in}{aaaaaaaaaaaaaaaaaaaaaaaaaaaaaaaaaaaaaaaaajjjjjjjjjjjjjjjjjjjjjjjjjjjjjjj
}}
      
      \caption{indcutancia de nose 
      magnetica no se que es f3}
      \label{figurelabel}
   \end{figure}
   



\section{CONCLUCIONES}

concluciones dponer

\addtolength{\textheight}{-12cm}  


%%%%%%%%%%%%%%%%%%%%%%%%%%%%%%%%%%%%%%%%%%%%%%%%%%%%%%%%%%%%%%%%%%%%%%%%%%%%%%%%



%%%%%%%%%%%%%%%%%%%%%%%%%%%%%%%%%%%%%%%%%%%%%%%%%%%%%%%%%%%%%%%%%%%%%%%%%%%%%%%%



%%%%%%%%%%%%%%%%%%%%%%%%%%%%%%%%%%%%%%%%%%%%%%%%%%%%%%%%%%%%%%%%%%%%%%%%%%%%%%%%
\section*{APPENDIX}

Appendixes should appear before the acknowledgment.

\section*{ACKNOWLEDGMENT}

las paginas de este libro bla bña.



%%%%%%%%%%%%%%%%%%%%%%%%%%%%%%%%%%%%%%%%%%%%%%%%%%%%%%%%%%%%%%%%%%%%%%%%%%%%%%%%

las referencias mas importantes



\begin{thebibliography}{99}

\bibitem{c1} G. O. Young, ÒSynthetic structure of industrial plastics (Book style with paper title and editor),Ó 	in Plastics, 2nd ed. vol. 3, J. Peters, Ed.  New York: McGraw-Hill, 1964, pp. 15Ð64.
\bibitem{c2} W.-K. Chen, Linear Networks and Systems (Book style).	Belmont, CA: Wadsworth, 1993, pp. 123Ð135.
\bibitem{c3} H. Poor, An Introduction to Signal Detection and Estimation.   New York: Springer-Verlag, 1985, ch. 4.
\bibitem{c4} B. Smith, ÒAn approach to graphs of linear forms (Unpublished work style),Ó unpublished.
\bibitem{c5} E. H. Miller, ÒA note on reflector arrays (Periodical styleÑAccepted for publication),Ó IEEE Trans. Antennas Propagat., to be publised.







\end{thebibliography}




\end{document}
