%%%%%%%%%%%%%%%%%%%%%%%%%%%%%%%%%%%%%%%%%%%%%%%%%%%%%%%%%%%%%%%%%%%%%%%%%%%%%%%%
%2345678901234567890123456789012345678901234567890123456789012345678901234567890
%        1         2         3         4         5         6         7         8

\documentclass[journal]{IEEEtran}
\hyphenation{op-tical net-works semi-conduc-tor}
\documentclass[letterpaper, 10 pt, conference]{ieeeconf} 
\IEEEoverridecommandlockouts                
\overrideIEEEmargins

\title{\LARGE \bf Aplicaci\'on de Estructuras de Datos Probabilistico en
}

\author{Samuel Leiva$^{1}$, Junior Micha$^{2}$ , Danilo Blas$^{3}$ Joel Janampa$^{4}$, Brener Bustillos$^{5}$% <-this % stops a space
\thanks{Manuscrito creado el 11 de setiembre del 2018; cuya revisio\'n final sera el 363 de dicimebre .Este trabajo es compatible en formato IEEE y se distribuye bajo el Proyecto LaTeX.El manuscrito puede ser encontrado en los github de los autores}% <-this % stops a space
\thanks{$^{1}$ S.Leiva es estudiante de ciencias de la computacion 
        Universidad Nacional de Ingenieria,2015-2021, Lima,Peru.
        {\tt\small https://github.com/SamuelLeiva}}%
\thanks{$^{2}$ J.Micha es estudiante de pregrado de matematica ,Universidad Nacional de Ingenieria,2016-2022,Lima,Peru.
        {\tt\small https://github.com/JMicha23}}%
\thanks{$^{3}$ D.Blas es estudiante de ciencias de la computacion ,Universidad Nacional de Ingenieria,2015-2021,Lima,Peru.
        {\tt\small https://github.com/Sdann26}}%
\thanks{$^{4}$ J.Janampa es estudiante de matematica,Universidad Nacional de Ingenieria,2015-2021,Lima,Peru.
        {\tt\small https://github.com/JoelJanampaBautista}}%
\thanks{$^{5}$ B.Bustillos es estudiante de matematica,Universidad Nacional de Ingenieria,Universidad Nacional de Ingenieria,Lima,Peru. 
        {\tt\small https://github.com/brenner-08}}%
}


\begin{document}



\maketitle
\thispagestyle{empty}
\pagestyle{empty}


%%%%%%%%%%%%%%%%%%%%%%%%%%%%%%%%%%%%%%%%%%%%%%%%%%%%%%%%%%%%%%%%%%%%%%%%%%%%%%%%
\begin{abstract}
    

Este articulo matemático-computacional presenta los resultados obtenidos mediante un proceso de investigacion sobre algunas estructuras de datos alternativas que pueden hacer que el trabajo de analisis de datos sea un más eficaz, antes de usar tecnicas como un clúster de cómputo para ejecutar herramientas distribuidas de cómputo paralelo como por ejemplo  Hadoop y Spark que son herramientas correctas ,pero muy costosas.

\end{abstract}


%%%%%%%%%%%%%%%%%%%%%%%%%%%%%%%%%%%%%%%%%%%%%%%%%%%%%%%%%%%%%%%%%%%%%%%%%%%%%%%%
\section{INTRODUCCIO\'N}

usar la seccion para poner 

\section{ESTADO DEL ARTE}

\begin{itemize}

\item \textbf{"Theory and Practice of Bloom Filters for Distributed Systems"}

Este articulo presenta una serie de técnicas probabilísticas como los Bloom-filters y sus variantes como stable Bloom Filter, Adaptative Bloom Filters, Filter Banks, etc. que se utilizan para reducir el procedimiento de la información y los costos de información.\\

\item \textbf{"An Improved Data Stream Summary: The Count-Min Sketch and its Applications"}

Este articulo presenta otras aplicaciones de la estructura de datos Count-min Sketch para problemas como encontrar cuartiles, elementos frecuentes, etc.\\

\item \textbf{"HyperLogLog in Practice: Algorithmic Engineering of a State of The Art Cardinality Estimation Algorithm"}

En este articulo se presentan mejoras al algoritmo HyperLogLog reduciendo sus requisitos de memoria y aumentar su precisión para un rango importante de cardinalidades.\\

\item \bf"Falta UNO" 

\end{itemize}


\section{OBJETIVOS}


\subsection{JSJSJS}

\subsection{UnDSDFs}

\begin{itemize}

\item URWER 
\item AERWE
\item DFSDF
\item DDD, 

\end{itemize}


\subsection{Ecuaciones}

sdajflas ecuanciones

$$
\alpha + \beta = \chi \eqno{(1)}
$$

Note that the equation is centered using a 

\subsection{subitituo}
\begin{itemize}


\item 
\item 
\item  

\end{itemize}


\section{INTRODUCCION}

RELLENAR 

\subsection{PIRMER}



\subsection{LLENAR SI SE DESEA}



   \begin{figure}[thpb]
      \centering
      \framebox{\parbox{3in}{wiiiiisdshdfksbsdka
}}
      %\includegraphics[scale=1.0]{figurefile}
      \caption{descripcion de la imagen }
      \label{figurelabel}
   \end{figure}
   


\section{CONCLUSIONS}



\addtolength{\textheight}{-12cm}


%%%%%%%%%%%%%%%%%%%%%%%%%%%%%%%%%%%%%%%%%%%%%%%%%%%%%%%%%%%%%%%%%%%%%%%%%%%%%%%%

%%%%%%%%%%%%%%%%%%%%%%%%%%%%%%%%%%%%%%%%%%%%%%%%%%%%%%%%%%%%%%%%%%%%%%%%%%%%%%%%

%%%%%%%%%%%%%%%%%%%%%%%%%%%%%%%%%%%%%%%%%%%%%%%%%%%%%%%%%%%%%%%%%%%%%%%%%%%%%%%%
\section*{APPENDIX}

rellenear

\section*{ACKNOWLEDGMENT}

%%%%%%%%%%%%%%%%%%%%%%%%%%%%%%%%%%%%%%%%%%%%%%%%%%%%%%%%%%%%%%%%%%%%%%%%%%%%%%%%
\begin{thebibliography}{99}

\bibitem{c1}
\bibitem{c2} 
\bibitem{c3} 


\end{thebibliography}

\end{document}
