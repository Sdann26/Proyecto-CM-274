%%%%%%%%%%%%%%%%%%%%%%%%%%%%%%%%%%%%%%%%%%%%%%%%%%%%%%%%%%%%%%%%%%%%%%%%%%%%%%%%
%2345678901234567890123456789012345678901234567890123456789012345678901234567890
%        1         2         3         4         5         6         7         8

\documentclass[journal]{IEEEtran}
\hyphenation{op-tical net-works semi-conduc-tor}
\documentclass[letterpaper, 10 pt, conference]{ieeeconf} 
\IEEEoverridecommandlockouts                
\overrideIEEEmargins

\title{\LARGE \bf Aplicaci\'on de Estructuras de Datos Probabilistico en
}

\author{Samuel Leiva$^{1}$, Junior Micha$^{2}$ , Danilo Blas$^{3}$ Joel Janampa$^{4}$, Brener Bustillos$^{5}$% <-this % stops a space
\thanks{Manuscrito creado el 11 de setiembre del 2018; cuya revisio\'n final sera el 363 de dicimebre .Este trabajo es compatible en formato IEEE y se distribuye bajo el Proyecto LaTeX.El manuscrito puede ser encontrado en los github de los autores}% <-this % stops a space
\thanks{$^{1}$ S.Leiva es estudiante de ciencias de la computacion 
        Universidad Nacional de Ingenieria,2015-2021, Lima,Peru.
        {\tt\small https://github.com/SamuelLeiva}}%
\thanks{$^{2}$ J.Micha es estudiante de pregrado de matematica ,Universidad Nacional de Ingenieria,2016-2022,Lima,Peru.
        {\tt\small https://github.com/JMicha23}}%
\thanks{$^{3}$ D.Blas es estudiante de ciencias de la computacion ,Universidad Nacional de Ingenieria,2015-2021,Lima,Peru.
        {\tt\small https://github.com/Sdann26}}%
\thanks{$^{4}$ J.Janampa es estudiante de matematica,Universidad Nacional de Ingenieria,2015-2021,Lima,Peru.
        {\tt\small https://github.com/JoelJanampaBautista}}%
\thanks{$^{5}$ B.Bustillos es estudiante de matematica,Universidad Nacional de Ingenieria,Universidad Nacional de Ingenieria,Lima,Peru. 
        {\tt\small https://github.com/brenner-08}}%
}


\begin{document}



\maketitle
\thispagestyle{empty}
\pagestyle{empty}


%%%%%%%%%%%%%%%%%%%%%%%%%%%%%%%%%%%%%%%%%%%%%%%%%%%%%%%%%%%%%%%%%%%%%%%%%%%%%%%%
\begin{abstract}
    

Este articulo matemático-computacional presenta los resultados obtenidos mediante un proceso de investigacion sobre algunas estructuras de datos alternativas que pueden hacer que el trabajo de analisis de datos sea un más eficaz, antes de usar tecnicas como un clúster de cómputo para ejecutar herramientas distribuidas de cómputo paralelo como por ejemplo  Hadoop y Spark que son herramientas correctas ,pero muy costosas.

\end{abstract}


%%%%%%%%%%%%%%%%%%%%%%%%%%%%%%%%%%%%%%%%%%%%%%%%%%%%%%%%%%%%%%%%%%%%%%%%%%%%%%%%
\section{INTRODUCCIO\'N}

usar la seccion para poner 

\section{RESUMEN}
Usar la estructura Bloom Filter para hacer consultas sobre elemntos que puedan pertenecer a la estructura.\\
usar la estructura count-min sketch 

\section{OBJETIVOS}


\subsection{JSJSJS}

\subsection{UnDSDFs}

\begin{itemize}

\item URWER 
\item AERWE
\item DFSDF
\item DDD, 

\end{itemize}


\subsection{Ecuaciones}

sdajflas ecuanciones

$$
\alpha + \beta = \chi \eqno{(1)}
$$

Note that the equation is centered using a 

\subsection{subitituo}
\begin{itemize}


\item 
\item 
\item  

\end{itemize}


\section{INTRODUCCION}

RELLENAR 

\subsection{PIRMER}



\subsection{LLENAR SI SE DESEA}



   \begin{figure}[thpb]
      \centering
      \framebox{\parbox{3in}{wiiiiisdshdfksbsdka
}}
      %\includegraphics[scale=1.0]{figurefile}
      \caption{descripcion de la imagen }
      \label{figurelabel}
   \end{figure}
   


\section{CONCLUSIONS}



\addtolength{\textheight}{-12cm}


%%%%%%%%%%%%%%%%%%%%%%%%%%%%%%%%%%%%%%%%%%%%%%%%%%%%%%%%%%%%%%%%%%%%%%%%%%%%%%%%

%%%%%%%%%%%%%%%%%%%%%%%%%%%%%%%%%%%%%%%%%%%%%%%%%%%%%%%%%%%%%%%%%%%%%%%%%%%%%%%%

%%%%%%%%%%%%%%%%%%%%%%%%%%%%%%%%%%%%%%%%%%%%%%%%%%%%%%%%%%%%%%%%%%%%%%%%%%%%%%%%
\section*{APPENDIX}

rellenear

\section*{ACKNOWLEDGMENT}

%%%%%%%%%%%%%%%%%%%%%%%%%%%%%%%%%%%%%%%%%%%%%%%%%%%%%%%%%%%%%%%%%%%%%%%%%%%%%%%%
\begin{thebibliography}{99}

\bibitem{c1}
\bibitem{c2} 
\bibitem{c3} 


\end{thebibliography}

\end{document}
